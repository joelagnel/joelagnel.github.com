%% start of file `template.tex'.
%% Copyright 2006-2010 Xavier Danaux (xdanaux@gmail.com).
%
% This work may be distributed and/or modified under the
% conditions of the LaTeX Project Public License version 1.3c,
% available at http://www.latex-project.org/lppl/.


\documentclass[11pt,a4paper]{moderncv}

% moderncv themes
\moderncvtheme[blue]{classic}                 % optional argument are 'blue' (default), 'orange', 'red', 'green', 'grey' and 'roman' (for roman fonts, instead of sans serif fonts)
%\moderncvtheme[green]{classic}                % idem
% character encoding
\usepackage[utf8]{inputenc}                   % replace by the encoding you are using

% adjust the page margins
\usepackage[scale=0.81]{geometry}

%\setlength{\hintscolumnwidth}{3cm}						% if you want to change the width of the column with the dates
%\AtBeginDocument{\setlength{\maketitlenamewidth}{6cm}}  % only for the classic theme, if you want to change the width of your name placeholder (to leave more space for your address details
%\AtBeginDocument{\recomputelengths}                     % required when changes are made to page layout lengths

% personal data
\firstname{Joel Agnel}
\familyname{Fernandes}
\title{Joel Agnel Fernandes}               % optional, remove the line if not wanted
\address{11911 Greenville Avenue, Apt 3125}{Dallas, TX 75243}    % optional, remove the line if not wanted
\mobile{650-450-8135}                    % optional, remove the line if not wanted
\email{agnel.joel@gmail.com}                      % optional, remove the line if not wanted
\homepage{http://www.linuxinternals.org/}                % optional, remove the line if not wanted
\extrainfo{\small{\url{http://linkedin.com/in/joelagnel}}}
% \extrainfo{Systems Engineer, Kernel Developer and Researcher} % optional, remove the line if not wanted
%\photo[64pt]{picture}                         % '64pt' is the height the picture must be resized to and 'picture' is the name of the picture file; optional, remove the line if not wanted
% \quote{RESUME}                 % optional, remove the line if not wanted

% to show numerical labels in the bibliography; only useful if you make citations in your resume
\makeatletter
\renewcommand*{\bibliographyitemlabel}{\@biblabel{\arabic{enumiv}}}
\makeatother

% bibliography with mutiple entries
%\usepackage{multibib}
%\newcites{book,misc}{{Books},{Others}}

%\nopagenumbers{}                             % uncomment to suppress automatic page numbering for CVs longer than one page
%----------------------------------------------------------------------------------
%            content
%----------------------------------------------------------------------------------
%\renewcommand{\section}[1]{\textsc{#1}}
\begin{document}
\maketitle
\section{Objective}
\cvline{}{To obtain a challenging kernel engineering position where I can apply my skills in my areas of expertise including:}
\cvlistitem{In-depth Knowledge of ARM Architecture.}
\cvlistitem{In-depth Knowledge of Memory models, Caching, Barriers and DMA.}
\cvlistitem{Successful bring up of complex SoC on FPGA and real boards.}
\cvlistitem{Expertise in writing Device Drivers for Linux kernel.}
\cvlistitem{Expertise in writing Parallel code for Multi-core systems.}
\cvlistitem{Knowledge of peripherals such as i2c, SPI, Timers, DMA controllers.}
\cvlistitem{Exploring and writing about OS internals (www.linuxinternals.org).}

\section{Technical Skills}
\cvline{\textsc{Programming}}{\small C, Assembly, Embedded C, Lisp, Ruby, Python, Perl, Erlang, \LaTeX, Verilog.}
\cvline{\textsc{Processors}}{\small ARM Cortex-A, Cortex-M, ARM9, Amber core, MSP430, x86, x86-64, 8051.}
\cvline{\textsc{OS}}{\small Linux, Android, ucLinux, Windows.}
\cvline{\textsc{Debug Tools}}{\small GDB, KGDB, KDB, Ftrace, SystemTap, ktap, Lauterbach, OpenOCD, Eclipse, Wireshark. }
\cvline{\textsc{Other Tools}}{\small Git, SVN, Vim, Emacs, cscope, Gnuplot, Matlab, ModelSim.}
\small{\newline}
\newline
\newline
\section{Work Experience}
\cventry{Jun 2012 - Present}{Senior Systems Engineer}{Linux Core Product Development, Texas Instruments}{}{}
{Embedded Systems Design, Development focusing on processors, early kernel boot, DMA and security. Active contributor to several open source and open hardware projects including:
\begin{itemize}
\item \textbf{Linux Kernel}: Maintainer of the EDMA DMA Engine driver. Authored several improvements to DMA (performance and framework). Author of OMAP DES driver for OMAP SoCs. Also, I've worked on every level of the kernel stack including machine layer, early boot code (ARM), block layer, file systems, networking and display drivers. Optimized performance and fixed bugs with heavy use of tracing, profile and debug tools. Proven track record to understand and take ownership of complex code in small amount of time and improve them.
\item \textbf{U-boot}: Contributed various features and bug fixes for TI OMAP SoCs. Modified U-boot to perform secure boot on SoCs. Improved U-boot's build system for first stage loader (SPL). Improved code density of binaries by analyzing Disassembly and conducting relevant experiments.
\item \textbf{OpenHardware development}: Baseport and board bring up (Linux Kernel and U-boot) for beagleboard.org boards. Developed hardware prototypes of different projects for Beagle community. Currently responsible for core Linux kernel support for TI's Davinci, OMAP and Sitara line of ARM based SoCs.
\end{itemize}}
\newline
\cventry{Jun 2011 - December 2011}{Embedded Systems Engineer Intern}{ARM MPU Business Unit, Texas Instruments}{}{}
{Software designer and architect of Beagleboard and Beagleboard-xM projects. Developed and debugged Display, Audio, USB, Networking, ADC subsystem drivers. Also, contributed support for Beagleboard-xM to U-boot mainline community.}

\cventry{September 2010 - May 2011}{\small Research Assistant}{\small Distributed Systems Lab, University of Texas at Dallas}{}{}
{Conducted research and experiments on different metrics to estimate the link conditions and thus improve routing. Modified drivers and mac80211 Linux Kernel code and carried out experiments. Worked on TCP/IP, UDP and other areas. Fixed bugs in wireless driver and packet routing/bridging code.}

\cventry{2008--2010}{Linux Kernel Developer}{Atlantis Computing}{Bangalore}{}{Responsible for design, development and maintenance of block layer and file system components in the virtualization product stack.\newline
\emph{Components developed:}
\begin{itemize}
\item \textbf{dedup-fs}: A data deduplication layer in the linux kernel to deduplicate redundant blocks in the ext2/ext3 file systems. Also implemented a parallel lazy-dedup version of the same for soft real time data deduplication that scales on multi-core architectures.
\item \textbf{dm-cache}:  Improved linux kernel device-mapper's dm-cache module with the following enhancements:
	\begin{enumerate}
		\item most-frequently-used (MFU) content-aware caching mechanism.
		\item Periodic write-back of dirty-blocks from cache to improve cache effectiveness.
	\end{enumerate}
\end{itemize}}
\newline
\cventry{2007--2008}{Embedded Systems Engineer}{Siemens Information Systems Ltd.}{Bangalore}{}{Embedded Systems Developer in the Engine Management Systems team.\newline
Worked on development of safety critical monitoring software layer using  MSP 430 low-power microcontroller, Embedded C and Python.}

\section{Academics}
\subsection{Education and Achievements}
\cventry{2010--2012}{MS in Computer Engineering}{University of Texas at Dallas}{}{}{
\begin{itemize}
	\item Obtained Grade Point Average (GPA) of 3.97/4.0. "A" Grade in 10/11 courses.
	\item Awarded full tuition waiver scholarship by the department for Spring 2011.
\end{itemize}}
\cventry{2003--2007}{BE in Electronics and Communication}{Visvesvaraya Technological University}{}{}{
\begin{itemize}
	\item First class with distinction in all semesters (1st to 8th sem), second rank in the college.
	\item Received Merit Scholarship during the 2nd and 3rd year (2004--2005 and 2005--2006).
\end{itemize}}

\subsection{\textsc{Research Interests}}
\cvlistdoubleitem{Realtime Systems}{Multicore processors}
\cvlistdoubleitem{OS Schedulers}{Concurrency}
\cvlistdoubleitem{Computer Architecture}{Cache Coherent Systems}

\subsection{\textsc{Relevant Coursework}}
\cvlistdoubleitem{Computer Architecture}{Operating Systems} 
\cvlistdoubleitem{Microprocessors}{VLSI Design} 
\cvlistdoubleitem{Advanced Operating Systems}{Computer Arithmetic}
\end{document}
